% Options for packages loaded elsewhere
\PassOptionsToPackage{unicode}{hyperref}
\PassOptionsToPackage{hyphens}{url}
%
\documentclass[
]{article}
\usepackage{amsmath,amssymb}
\usepackage{lmodern}
\usepackage{iftex}
\ifPDFTeX
  \usepackage[T1]{fontenc}
  \usepackage[utf8]{inputenc}
  \usepackage{textcomp} % provide euro and other symbols
\else % if luatex or xetex
  \usepackage{unicode-math}
  \defaultfontfeatures{Scale=MatchLowercase}
  \defaultfontfeatures[\rmfamily]{Ligatures=TeX,Scale=1}
\fi
% Use upquote if available, for straight quotes in verbatim environments
\IfFileExists{upquote.sty}{\usepackage{upquote}}{}
\IfFileExists{microtype.sty}{% use microtype if available
  \usepackage[]{microtype}
  \UseMicrotypeSet[protrusion]{basicmath} % disable protrusion for tt fonts
}{}
\makeatletter
\@ifundefined{KOMAClassName}{% if non-KOMA class
  \IfFileExists{parskip.sty}{%
    \usepackage{parskip}
  }{% else
    \setlength{\parindent}{0pt}
    \setlength{\parskip}{6pt plus 2pt minus 1pt}}
}{% if KOMA class
  \KOMAoptions{parskip=half}}
\makeatother
\usepackage{xcolor}
\IfFileExists{xurl.sty}{\usepackage{xurl}}{} % add URL line breaks if available
\IfFileExists{bookmark.sty}{\usepackage{bookmark}}{\usepackage{hyperref}}
\hypersetup{
  hidelinks,
  pdfcreator={LaTeX via pandoc}}
\urlstyle{same} % disable monospaced font for URLs
\setlength{\emergencystretch}{3em} % prevent overfull lines
\providecommand{\tightlist}{%
  \setlength{\itemsep}{0pt}\setlength{\parskip}{0pt}}
\setcounter{secnumdepth}{-\maxdimen} % remove section numbering
\ifLuaTeX
  \usepackage{selnolig}  % disable illegal ligatures
\fi

\author{}
\date{}

\begin{document}

\textbf{Problem Statement: Physics Game - Collisions and Gravity}

Develop a physics-based gaming application using C++/C\# or JavaScript
that focuses on realistic collision dynamics and gravitational
interactions. The goal is to create an engaging gameplay experience
where players can manipulate objects, analyze trajectories, and master
the application of physics principles in a challenging gaming
environment.

\textbf{Showcasing:}

realistic collision dynamics, gravitational forces, and the innovative
incorporation of momentum.

\textbf{Requirements:}

\begin{enumerate}
\def\labelenumi{\arabic{enumi}.}
\item
  \textbf{Game Elements:}

  \begin{itemize}
  \item
    \textbf{Projectile:} Represented as an object (e.g., bird,
    character) with an initial position, velocity, and mass.
  \item
    \textbf{Targets/Obstacles:} Include multiple targets or obstacles
    with different shapes and sizes.
  \end{itemize}
\item
  \textbf{Physics Simulation:}

  \begin{itemize}
  \item
    Implement realistic collision interactions between rigid bodies
    (Projectile and Targets/Obstacles) based on initial conditions.
  \item
    Integrate gravitational forces that influence the trajectory and
    motion of the rigid bodies.
  \item
    Consider additional forces (e.g., wind) to introduce complexity and
    challenge.
  \item
    Provide a physics engine that takes specified inputs and calculates
    the new configuration of the scene for rigid bodies.
  \end{itemize}
\item
  \textbf{Physics inputs:}
\end{enumerate}

\begin{itemize}
\item
  Define initial conditions for the scene, assuming rigid bodies. This
  includes specifying the locations of bodies (Projectile,
  Targets/Obstacles), their initial velocity, and forces acting upon
  them.
\item
  Allow the specification of additional parameters such as launch angle,
  force, and time step size for simulation accuracy.
\end{itemize}

\begin{enumerate}
\def\labelenumi{\arabic{enumi}.}
\setcounter{enumi}{3}
\item
  \textbf{Physics Outputs:}

  \begin{itemize}
  \item
    Output the updated positions and velocities of the rigid bodies
    after each simulation step.
  \item
    Visualize the scene's new configuration, considering any changes
    resulting from rigid body collisions, gravitational effects, or
    other forces.
  \end{itemize}
\item
  \textbf{Controls:}
\end{enumerate}

\begin{itemize}
\item
  Enable user input controls for setting initial conditions and
  parameters.
\item
  Allow users to specify launch angle, force, and other relevant physics
  parameters for rigid bodies.
\item
  Implement controls for adjusting the time step size to control
  simulation accuracy.
\end{itemize}

\begin{enumerate}
\def\labelenumi{\arabic{enumi}.}
\setcounter{enumi}{5}
\item
  \textbf{Graphics and Sound:}

  \begin{itemize}
  \item
    Utilize graphics to visually represent the physics-based
    interactions in the game, emphasizing rigid body collisions and
    movements.
  \item
    Implement sound effects corresponding to key physics events such as
    rigid body collisions and launches.
  \end{itemize}
\end{enumerate}

\textbf{Project Goal}

The primary ambition of this project is to achieve a breakthrough in
physics-based computation within the gaming realm. By focusing on the
intricacies of realistic collision dynamics, gravitational interactions,
and pioneering the implementation of dynamic momentum disruptions, the
project aims to elevate the gaming experience to a level where players
actively engage with and manipulate fundamental physics principles. When
a bird collides, the goal is to authentically break the gravity
momentum, simulating a quantum leap in computational realism. This
project aspires to contribute to the advancement of physics
understanding in the gaming community, pushing the boundaries of what is
achievable in virtual worlds through cutting-edge computational physics.

\textbf{Key Implementation Guidelines:}

\begin{itemize}
\item
  Develop classes and functions to manage physics inputs and outputs for
  rigid bodies.
\item
  Implement a physics engine that accurately models collision dynamics
  and gravitational interactions for rigid bodies.
\item
  Provide a user-friendly interface for adjusting physics parameters.
\item
  Utilize graphics libraries or frameworks to visualize the
  physics-based interactions of rigid bodies.
\item
  Implement sound effects corresponding to key physics events involving
  rigid bodies.
\end{itemize}

\end{document}
