\documentclass{article}
\usepackage{tabularx}
\usepackage{booktabs}

\title{Physics Game - Collisions and Gravity: Problem Statement}
\author{} % Add author name here
\date{} % Add date here

\begin{document}

\maketitle

\section{Problem Statement}
Develop a physics-based game focusing on realistic collision dynamics and gravitational interactions using C++/C\# or JavaScript. The aim is to engage players in manipulating objects, analyzing trajectories, and applying physics principles in a challenging environment.

\section{Game Elements}
\begin{itemize}
  \item Projectile: Object with initial position, velocity, and mass.
  \item Targets/Obstacles: Various shapes and sizes.
\end{itemize}

\section{Physics Simulation}
\begin{itemize}
  \item Realistic collision interactions.
  \item Gravity affecting projectile motion.
  \item Additional forces like wind.
  \item Visual feedback for collisions.
\end{itemize}

\section{Controls}
\begin{itemize}
  \item Intuitive controls (mouse clicks, touch inputs).
  \item Adjustable launch angle, force, etc.
\end{itemize}

\section{Graphics and Sound}
\begin{itemize}
  \item Enhanced graphics.
  \item Complementary sound effects.
\end{itemize}

\section{Goals}
\begin{itemize}
  \item Engage players in a physics-based interactive environment.
  \item Demonstrate realistic physics phenomena.
\end{itemize}

\section{Stretch Goals}
\begin{itemize}
  \item Enhance physics simulations for more complexity.
  \item Expand control options for diverse gameplay.
  \item Upgrade graphics and sound for a more immersive experience.
\end{itemize}

\end{document}
