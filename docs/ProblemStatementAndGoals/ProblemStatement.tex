% Options for packages loaded elsewhere
\PassOptionsToPackage{unicode}{hyperref}
\PassOptionsToPackage{hyphens}{url}
%
\documentclass[
]{article}
\usepackage{amsmath,amssymb}
\usepackage{lmodern}
\usepackage{iftex}
\ifPDFTeX
  \usepackage[T1]{fontenc}
  \usepackage[utf8]{inputenc}
  \usepackage{textcomp} % provide euro and other symbols
\else % if luatex or xetex
  \usepackage{unicode-math}
  \defaultfontfeatures{Scale=MatchLowercase}
  \defaultfontfeatures[\rmfamily]{Ligatures=TeX,Scale=1}
\fi
% Use upquote if available, for straight quotes in verbatim environments
\IfFileExists{upquote.sty}{\usepackage{upquote}}{}
\IfFileExists{microtype.sty}{% use microtype if available
  \usepackage[]{microtype}
  \UseMicrotypeSet[protrusion]{basicmath} % disable protrusion for tt fonts
}{}
\makeatletter
\@ifundefined{KOMAClassName}{% if non-KOMA class
  \IfFileExists{parskip.sty}{%
    \usepackage{parskip}
  }{% else
    \setlength{\parindent}{0pt}
    \setlength{\parskip}{6pt plus 2pt minus 1pt}}
}{% if KOMA class
  \KOMAoptions{parskip=half}}
\makeatother
\usepackage{xcolor}
\IfFileExists{xurl.sty}{\usepackage{xurl}}{} % add URL line breaks if available
\IfFileExists{bookmark.sty}{\usepackage{bookmark}}{\usepackage{hyperref}}
\hypersetup{
  hidelinks,
  pdfcreator={LaTeX via pandoc}}
\urlstyle{same} % disable monospaced font for URLs
\usepackage{longtable,booktabs,array}
\usepackage{calc} % for calculating minipage widths
% Correct order of tables after \paragraph or \subparagraph
\usepackage{etoolbox}
\makeatletter
\patchcmd\longtable{\par}{\if@noskipsec\mbox{}\fi\par}{}{}
\makeatother
% Allow footnotes in longtable head/foot
\IfFileExists{footnotehyper.sty}{\usepackage{footnotehyper}}{\usepackage{footnote}}
\makesavenoteenv{longtable}
\setlength{\emergencystretch}{3em} % prevent overfull lines
\providecommand{\tightlist}{%
  \setlength{\itemsep}{0pt}\setlength{\parskip}{0pt}}
\setcounter{secnumdepth}{-\maxdimen} % remove section numbering
\ifLuaTeX
  \usepackage{selnolig}  % disable illegal ligatures
\fi

\author{}
\date{}

\begin{document}

\textbf{Problem Statement and Goals}

\textbf{Physics Game - Collisions and Gravity}

Al Jubair Hossain

Feb 4, 2024

Table 1: Goal statement

\begin{longtable}[]{@{}
  >{\raggedright\arraybackslash}p{(\columnwidth - 2\tabcolsep) * \real{0.4999}}
  >{\raggedright\arraybackslash}p{(\columnwidth - 2\tabcolsep) * \real{0.5001}}@{}}
\toprule
\begin{minipage}[b]{\linewidth}\raggedright
\textbf{Date}
\end{minipage} & \begin{minipage}[b]{\linewidth}\raggedright
\textbf{Developer(s)}
\end{minipage} \\
\midrule
\endhead
Feb 4, 2024 & Al Jubair Hossain \\
\bottomrule
\end{longtable}

\begin{enumerate}
\def\labelenumi{\arabic{enumi}.}
\item
  \textbf{Problem Statement}
\end{enumerate}

Develop a physics-based gaming application using C++/C\# or JavaScript
that focuses on realistic collision dynamics and gravitational
interactions. The goal is to create an engaging gameplay experience
where players can manipulate objects, analyze trajectories, and master
the application of physics principles in a challenging gaming
environment.

\textbf{Showcasing:}

realistic collision dynamics, gravitational forces, and the innovative
incorporation of momentum.

\textbf{1.1 Problem}

Develop a physics-based gaming application focusing on realistic
collision dynamics, gravitational interactions, and innovative momentum
disruptions. The goal is to create an engaging gameplay experience
allowing players to manipulate objects, analyze trajectories, and master
physics principles in a challenging gaming environment.

\textbf{1.2 Inputs and Outputs}

\textbf{Inputs:}

\begin{itemize}
\item
  Initial Conditions: User-specified parameters for rigid bodies
  (Projectile, Targets/Obstacles).
\item
  Launch Parameters: User-defined launch angle, force, and time step
  size for simulation accuracy.
\end{itemize}

\textbf{Outputs:}

\begin{itemize}
\item
  Updated Positions and Velocities: Information on the new configuration
  of rigid bodies post each simulation step.
\item
  Visual Representation: Graphics illustrating the scene's new
  configuration, considering collisions, gravitational effects, and
  other forces.
\end{itemize}

\textbf{Project Goal}

\begin{enumerate}
\def\labelenumi{\arabic{enumi}.}
\item
  \textbf{Realistic Interaction Mastery:}
\end{enumerate}

\begin{quote}
Enable players to master realistic collision dynamics, gravitational
forces, and momentum disruptions, providing an immersive experience in
manipulating fundamental physics principles within the gaming
environment.
\end{quote}

\begin{enumerate}
\def\labelenumi{\arabic{enumi}.}
\setcounter{enumi}{1}
\item
  \textbf{Engaging Gameplay Experience:}
\end{enumerate}

\begin{quote}
Create an engaging and challenging gaming experience where players can
analyze trajectories, strategically manipulate objects, and actively
apply physics knowledge to overcome obstacles.
\end{quote}

\begin{enumerate}
\def\labelenumi{\arabic{enumi}.}
\setcounter{enumi}{2}
\item
  \textbf{Quantum Leap in Computational Realism:}
\end{enumerate}

\begin{quote}
Simulate a breakthrough in computational realism by authentically
representing the impact of collisions, gravitational effects, and
dynamic momentum disruptions, especially when a bird collides,
delivering a quantum leap in gaming physics.
\end{quote}

\begin{enumerate}
\def\labelenumi{\arabic{enumi}.}
\setcounter{enumi}{3}
\item
  \textbf{Advancement in Gaming Physics Understanding:}
\end{enumerate}

\begin{quote}
Contribute to the advancement of gaming physics understanding within the
community, pushing the boundaries of what is achievable in virtual
worlds through the application of cutting-edge computational physics.
\end{quote}

\begin{enumerate}
\def\labelenumi{\arabic{enumi}.}
\setcounter{enumi}{4}
\item
  \textbf{Elevated User Engagement:}
\end{enumerate}

\begin{quote}
Elevate user engagement by providing a gaming experience that not only
entertains but also educates, encouraging players to actively engage
with and deepen their understanding of physics principles through
interactive gameplay.
\end{quote}

\textbf{3. Stretch Goals}

\begin{enumerate}
\def\labelenumi{\arabic{enumi}.}
\item
  \textbf{Advanced Physics Interactions:} Explore implementing more
  complex physics phenomena, such as fluid dynamics or advanced particle
  interactions.
\item
  \textbf{Multiplayer Functionality:} Extend the application to support
  multiplayer interactions, enabling collaborative or competitive
  gameplay.
\item
  \textbf{Integration of AI Elements:} Investigate the incorporation of
  AI-driven entities to enhance the gaming environment's realism and
  challenge level.
\end{enumerate}

\end{document}
