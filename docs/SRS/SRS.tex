% Options for packages loaded elsewhere
\PassOptionsToPackage{unicode}{hyperref}
\PassOptionsToPackage{hyphens}{url}
%
\documentclass[
]{article}
\usepackage{amsmath,amssymb}
\usepackage{lmodern}
\usepackage{iftex}
\ifPDFTeX
  \usepackage[T1]{fontenc}
  \usepackage[utf8]{inputenc}
  \usepackage{textcomp} % provide euro and other symbols
\else % if luatex or xetex
  \usepackage{unicode-math}
  \defaultfontfeatures{Scale=MatchLowercase}
  \defaultfontfeatures[\rmfamily]{Ligatures=TeX,Scale=1}
\fi
% Use upquote if available, for straight quotes in verbatim environments
\IfFileExists{upquote.sty}{\usepackage{upquote}}{}
\IfFileExists{microtype.sty}{% use microtype if available
  \usepackage[]{microtype}
  \UseMicrotypeSet[protrusion]{basicmath} % disable protrusion for tt fonts
}{}
\makeatletter
\@ifundefined{KOMAClassName}{% if non-KOMA class
  \IfFileExists{parskip.sty}{%
    \usepackage{parskip}
  }{% else
    \setlength{\parindent}{0pt}
    \setlength{\parskip}{6pt plus 2pt minus 1pt}}
}{% if KOMA class
  \KOMAoptions{parskip=half}}
\makeatother
\usepackage{xcolor}
\IfFileExists{xurl.sty}{\usepackage{xurl}}{} % add URL line breaks if available
\IfFileExists{bookmark.sty}{\usepackage{bookmark}}{\usepackage{hyperref}}
\hypersetup{
  hidelinks,
  pdfcreator={LaTeX via pandoc}}
\urlstyle{same} % disable monospaced font for URLs
\usepackage{longtable,booktabs,array}
\usepackage{calc} % for calculating minipage widths
% Correct order of tables after \paragraph or \subparagraph
\usepackage{etoolbox}
\makeatletter
\patchcmd\longtable{\par}{\if@noskipsec\mbox{}\fi\par}{}{}
\makeatother
% Allow footnotes in longtable head/foot
\IfFileExists{footnotehyper.sty}{\usepackage{footnotehyper}}{\usepackage{footnote}}
\makesavenoteenv{longtable}
\setlength{\emergencystretch}{3em} % prevent overfull lines
\providecommand{\tightlist}{%
  \setlength{\itemsep}{0pt}\setlength{\parskip}{0pt}}
\setcounter{secnumdepth}{-\maxdimen} % remove section numbering
\ifLuaTeX
  \usepackage{selnolig}  % disable illegal ligatures
\fi

\author{}
\date{}

\begin{document}

\title{System Requirements Specification (SRS) for Physics-Based Game \progname{}} 
\author{\authname}
\date{\today}
	
\maketitle

\pagenumbering{roman}

\newpage

\textbf{Revision History}

\begin{longtable}[]{@{}
  >{\raggedright\arraybackslash}p{(\columnwidth - 4\tabcolsep) * \real{0.3332}}
  >{\raggedright\arraybackslash}p{(\columnwidth - 4\tabcolsep) * \real{0.3334}}
  >{\raggedright\arraybackslash}p{(\columnwidth - 4\tabcolsep) * \real{0.3334}}@{}}
\toprule
\begin{minipage}[b]{\linewidth}\raggedright
\textbf{Date}
\end{minipage} & \begin{minipage}[b]{\linewidth}\raggedright
\textbf{Version}
\end{minipage} & \begin{minipage}[b]{\linewidth}\raggedright
\textbf{Notes}
\end{minipage} \\
\midrule
\endhead
February 3, 2024 & 1 & Changed section (6, 7, 11) \\
February 26, 2024 & 2 & Changed section (1.4, 4.2.1, 4.4.2, 4.4.8) \\
\bottomrule
\end{longtable}

\newpage

\textbf{Table of Contents}

\

\textbf{\protect\hyperlink{qmm}{1. Reference Material}}

\protect\hyperlink{qs}{1.1 Table of Units}

\protect\hyperlink{qd}{1.2 Table of Symbols}

\protect\hyperlink{qf}{1.3 Abbreviations and Acronyms}

\protect\hyperlink{qg}{1.4 Mathematical Notation}

\

\textbf{\protect\hyperlink{qw}{2. Introduction}}

\protect\hyperlink{qww}{2.1 Purpose of Document}

\protect\hyperlink{qwww}{2.2 Scope of Requirements}

\protect\hyperlink{qwwww}{2.3 Characteristics of Intended Reader}

\

\protect\hyperlink{qe}{\textbf{3. General System Description}}

\protect\hyperlink{qee}{3.1 System Context}

\protect\hyperlink{qeee}{3.2 User Characteristics}

\protect\hyperlink{qeeee}{3.3 System Constraints}

\

\textbf{\protect\hyperlink{qr}{4. Specific System Description}}

\textbf{\protect\hyperlink{qrr}{4.1 Problem Description}}

\protect\hyperlink{qrrr}{4.1.1 Terminology and Definitions}

\protect\hyperlink{qrrrr}{4.1.2 Physical System Description}

\textbf{\protect\hyperlink{qrrrrrr}{4.2 Solution Characteristics
Specification}}

\protect\hyperlink{qrrrrrrr}{4.2.1 Types}

\textbf{\protect\hyperlink{qrrrrrrrr}{4.3 Scope Decisions}}

\textbf{\protect\hyperlink{qrrrrrrrrrrr}{4.4 Modelling Decisions}}

\protect\hyperlink{qrrrrrrrrrrrr}{4.4.1 Assumptions}

\protect\hyperlink{qrrrrrrrrrrrrrr}{4.4.2 Theoretical Models}

\protect\hyperlink{qrrrrrrrrrrrrrrr}{4.4.3 General Definitions}

\protect\hyperlink{qkl}{4.4.5 Data Types}

\protect\hyperlink{qrrrrrrrrrrrrrrrrr}{4.4.6 Instance Models}

\protect\hyperlink{qrrrrrrrrrrrrrrrrrr}{4.4.7 Input Data Constraints}

\protect\hyperlink{qrrrrrrrrrrrrrrrrrrr}{4.4.8 Properties of a Correct
Solution}

\

\textbf{\protect\hyperlink{qb}{5. Requirements}}

\protect\hyperlink{qbb}{5.1 Functional Requirements}

\protect\hyperlink{qbbb}{5.2 Nonfunctional Requirements}

\

\textbf{\protect\hyperlink{qn}{6. Likely Changes}}

\

\textbf{\protect\hyperlink{qm}{7. Unlikely Changes}}

\

\textbf{\protect\hyperlink{_Hlk157473991}{8. Traceability Matrices and
Graphs}}

\

\textbf{\protect\hyperlink{qzz}{9. Development Plan}}

\

\textbf{\protect\hyperlink{qzzz}{10. Values of Auxiliary Constants}}

\

\textbf{\protect\hyperlink{qop}{11. Project Goal}}

\

\textbf{\protect\hyperlink{qbn}{12. Goal in Physics}}

\newpage

\protect\hypertarget{qmm}{}{}\textbf{1. Reference Material}

\protect\hypertarget{qs}{}{}\textbf{1.1 Table of Units}

\begin{longtable}[]{@{}
  >{\raggedright\arraybackslash}p{(\columnwidth - 2\tabcolsep) * \real{0.5000}}
  >{\raggedright\arraybackslash}p{(\columnwidth - 2\tabcolsep) * \real{0.5000}}@{}}
\toprule
\begin{minipage}[b]{\linewidth}\raggedright
\textbf{Unit}
\end{minipage} & \begin{minipage}[b]{\linewidth}\raggedright
\textbf{Description}
\end{minipage} \\
\midrule
\endhead
m & Meter \\
sec & Second \\
kg & Kilogram \\
N & Newton \\
\bottomrule
\end{longtable}

\protect\hypertarget{qd}{}{}\textbf{1.2 Table of Symbols}

\begin{longtable}[]{@{}
  >{\raggedright\arraybackslash}p{(\columnwidth - 2\tabcolsep) * \real{0.5000}}
  >{\raggedright\arraybackslash}p{(\columnwidth - 2\tabcolsep) * \real{0.5000}}@{}}
\toprule
\begin{minipage}[b]{\linewidth}\raggedright
\textbf{Symbol}
\end{minipage} & \begin{minipage}[b]{\linewidth}\raggedright
\textbf{Description}
\end{minipage} \\
\midrule
\endhead
R & Position Vector \\
V & Velocity Vector \\
F & Force Vector \\
M & Mass \\
G & Gravitational acceleration vector \\
Θ & Launch angle \\
Fwind & Wind force \\
\bottomrule
\end{longtable}

\protect\hypertarget{qf}{}{}\textbf{1.3 Abbreviations and Acronyms}

\begin{longtable}[]{@{}
  >{\raggedright\arraybackslash}p{(\columnwidth - 2\tabcolsep) * \real{0.5000}}
  >{\raggedright\arraybackslash}p{(\columnwidth - 2\tabcolsep) * \real{0.5000}}@{}}
\toprule
\begin{minipage}[b]{\linewidth}\raggedright
\textbf{Abbreviation}
\end{minipage} & \begin{minipage}[b]{\linewidth}\raggedright
\textbf{Description}
\end{minipage} \\
\midrule
\endhead
SRS & Software requirement Specification \\
UI & User interface \\
API & Application Programming interface \\
FPS & Frames per second \\
\bottomrule
\end{longtable}

\protect\hypertarget{qg}{}{}\textbf{1.4 Mathematical Notation}

\begin{itemize}
\item
  v0: Initial velocity vector
\item
  r0: Initial position vector
\item
  Fnet : Net force vector
\item
  Δt : Time step size
\item
  a : Acceleration vector
\end{itemize}

\protect\hypertarget{qw}{}{}\textbf{2. Introduction}

\protect\hypertarget{qww}{}{}\textbf{2.1 Purpose of the Document}

The purpose of this Software Requirements Specification (SRS) document
is to provide a comprehensive and detailed outline of the requirements
for the development of a physics-based gaming application. It serves as
a guide for the development team, stakeholders, and any parties involved
in the project, ensuring a clear understanding of the features,
functionalities, and constraints of the proposed gaming application.

\protect\hypertarget{qwww}{}{}\textbf{2.2 Scope of Requirements}

The scope of this document encompasses the specifications and
requirements for the physics-based gaming application, emphasizing
realistic collision dynamics and gravitational interactions. It includes
the essential features such as projectile motion, collision handling,
gravitational forces, additional forces (e.g., wind), user controls,
graphics, and sound effects. The document also outlines the external
interfaces, non-functional requirements, and other aspects relevant to
the successful development of the gaming application.

\protect\hypertarget{qwwww}{}{}\textbf{2.3 Characteristics of Intended
Reader}

\begin{itemize}
\item
  \textbf{Development Team:} Software engineers, programmers, and
  designers involved in the implementation of the physics-based gaming
  application.
\item
  \textbf{Project Stakeholders:} Individuals or groups with a vested
  interest in the success of the project, such as project managers,
  quality assurance teams, and sponsors.
\item
  \textbf{Testing Team:} Professionals responsible for testing the
  application against specified requirements.
\item
  \textbf{End Users:} Gamers and individuals interested in understanding
  the technical aspects and functionalities of the gaming application.
\end{itemize}

\protect\hypertarget{qe}{}{}\textbf{3. General System Description:}

\protect\hypertarget{qee}{}{}\textbf{3.1 System Context}

The physics-based gaming application, as represented in the code,
operates within a console-based environment, interacting with the user
and simulating a simplified gaming experience.

\begin{itemize}
\item
  \textbf{External Systems:}

  \begin{itemize}
  \item
    \textbf{Console Environment:} The application relies on
    console-based input/output for user interaction and rendering game
    elements.
  \item
    \textbf{Graphics and Sound Libraries:} While the current
    implementation lacks explicit integration with external graphics and
    sound libraries, future development might benefit from incorporating
    these elements to enhance the gaming experience.
  \end{itemize}
\item
  \textbf{Users:}

  \begin{itemize}
  \item
    \textbf{Console Users (Gamers):}

    \begin{itemize}
    \item
      Interact with the game through keyboard inputs in the console.
    \item
      Experience a simplified representation of game elements through
      ASCII art.
    \item
      May be familiar with console-based games and basic keyboard
      controls.
    \end{itemize}
  \item
    \textbf{Developers:}

    \begin{itemize}
    \item
      Utilize C++ as the programming language for development.
    \item
      Require a console environment for testing and debugging.
    \item
      May enhance the application by integrating graphics and sound
      libraries in future iterations.
    \end{itemize}
  \end{itemize}
\end{itemize}

\protect\hypertarget{qeee}{}{}\textbf{3.2 User Characteristics}

\begin{itemize}
\item
  \textbf{Console Users (Gamers):}

  \begin{itemize}
  \item
    Comfortable with simple keyboard controls ('w', 'a', 'd', 'r', 'q')
    for gameplay.
  \item
    Familiar with console-based interactions.
  \item
    Seeking a basic, retro-style gaming experience.
  \end{itemize}
\item
  \textbf{Developers:}

  \begin{itemize}
  \item
    Proficient in C++ programming.
  \item
    Familiar with console-based development and debugging.
  \item
    May have the capability and interest in extending the application
    with more advanced features.
  \end{itemize}
\end{itemize}

\protect\hypertarget{qeeee}{}{}\textbf{3.3 System Constraints}

\begin{itemize}
\item
  \textbf{Hardware Constraints:}

  \begin{itemize}
  \item
    The application's performance is constrained by the capabilities of
    the console environment, limiting advanced graphical and audio
    features.
  \end{itemize}
\item
  \textbf{Physics Simulation Constraints:}

  \begin{itemize}
  \item
    The physics simulation is simplistic, suitable for a console-based
    environment but may lack the complexity of a full-fledged physics
    engine.
  \end{itemize}
\item
  \textbf{User Interface Constraints:}

  \begin{itemize}
  \item
    The user interface is confined to console-based interactions,
    restricting the visual representation and overall user experience.
  \end{itemize}
\end{itemize}

\protect\hypertarget{qr}{}{}\textbf{4. Specific System Description}

\protect\hypertarget{qrr}{}{}\textbf{4.1 Problem Description}

The physics-based gaming application aims to create an engaging and
challenging gaming experience where players can manipulate objects,
analyze trajectories, and apply physics principles within a dynamic and
interactive virtual environment. The primary focus is on realistic
collision dynamics, gravitational interactions, and the integration of
additional forces, such as wind.

\protect\hypertarget{qrrr}{}{}\textbf{4.1.1 Terminology and Definitions}

\begin{itemize}
\item
  \textbf{Projectile:} A game element, represented as an object (e.g.,
  bird, pig), possessing an initial position, velocity, and mass.
\item
  \textbf{Targets/Obstacles:} Various objects within the gaming
  environment, each having different shapes and sizes, serving as
  elements for interaction and challenge.
\item
  \textbf{Physics Simulation:} The process of modeling and simulating
  realistic collision interactions and gravitational forces between
  rigid bodies within the gaming environment.
\item
  \textbf{Gravitational Forces:} The influence of gravity on the
  trajectory and motion of rigid bodies, affecting their movement within
  the virtual space.
\item
  \textbf{Additional Forces (e.g., Wind):} Extra forces applied to rigid
  bodies to introduce complexity and challenge, enhancing the overall
  gameplay experience.
\item
  \textbf{Physics Engine:} The computational framework responsible for
  accurately modeling collision dynamics, gravitational interactions,
  and other physical phenomena within the gaming environment.
\end{itemize}

\protect\hypertarget{qrrrr}{}{}\textbf{4.1.2 Physical System
Description}

The physical system of the gaming application comprises virtual entities
that adhere to the principles of physics. These entities include:

\begin{itemize}
\item
  \textbf{Projectile:} An object with defined mass, initial position,
  and velocity, subject to the laws of motion and gravitational forces.
\item
  \textbf{Targets/Obstacles:} Diverse objects within the gaming
  environment, each with specific geometric characteristics, interacting
  with projectiles through collisions.
\item
  \textbf{Gravitational Field:} A virtual representation of gravity,
  influencing the trajectories of projectiles and contributing to the
  overall physics simulation.
\item
  \textbf{Additional Forces (e.g., Wind):} External forces that impact
  the motion of projectiles, introducing variability and complexity to
  the gameplay.
\end{itemize}

\begin{enumerate}
\def\labelenumi{\arabic{enumi}.}
\item
  \textbf{Scalability:} Design the system to be extensible for potential
  future expansions, allowing for the addition of new features, levels,
  and challenges.
\end{enumerate}

\protect\hypertarget{qrrrrrr}{}{}\textbf{4.2 Solution Characteristics
Specification}

The Solution Characteristics Specification outlines the key features and
types that define the characteristics of the physics-based gaming
application.

\protect\hypertarget{qrrrrrrr}{}{}\textbf{4.2.1 Types}

\textbf{4.2.1.1 Game Elements}

\begin{enumerate}
\def\labelenumi{\arabic{enumi}.}
\item
  \textbf{Projectile Type:}

  \begin{itemize}
  \item
    \emph{Example Instance:}

    \begin{itemize}
    \item
      Initial Position: (0, 0)
    \item
      Initial Velocity: (5, 5)
    \item
      Mass: 1.0
    \end{itemize}
  \end{itemize}
\item
  \textbf{Targets/Obstacles Type:}

  \begin{itemize}
  \item
    \emph{Example Instance:}

    \begin{itemize}
    \item
      Position: (10, 8)
    \item
      Shape: Rectangular
    \item
      Size: 3 units
    \end{itemize}
  \end{itemize}
\end{enumerate}

\textbf{4.2.1.2 Physics Simulation}

\begin{enumerate}
\def\labelenumi{\arabic{enumi}.}
\setcounter{enumi}{2}
\item
  \textbf{Collision Dynamics Type:}

  \begin{itemize}
  \item
    \emph{Example Instance:}

    \begin{itemize}
    \item
      Collision Detection Algorithm: Pixel-perfect collision
    \item
      Collision Response Algorithm: Destructive collision (objects break
      apart)
    \end{itemize}
  \end{itemize}
\item
  \textbf{Gravitational Forces Type:}

  \begin{itemize}
  \item
    \emph{Example Instance:}

    \begin{itemize}
    \item
      Gravitational Acceleration: 0 (ignoring gravity for simplicity)
    \end{itemize}
  \end{itemize}
\item
  \textbf{Additional Forces Type (e.g., Wind):}

  \begin{itemize}
  \item
    \emph{Example Instance:}

    \begin{itemize}
    \item
      Type of Force: Wind
    \item
      Magnitude and Direction: (2, 0) units/s² (horizontal)
    \end{itemize}
  \end{itemize}
\item
  \textbf{Physics Engine Type:}

  \begin{itemize}
  \item
    \emph{Example Instance:}

    \begin{itemize}
    \item
      Simulation Accuracy: Medium
    \item
      Computational Efficiency: Balanced
    \end{itemize}
  \end{itemize}
\end{enumerate}

\textbf{4.2.1.3 User Interaction}

\begin{enumerate}
\def\labelenumi{\arabic{enumi}.}
\setcounter{enumi}{6}
\item
  \textbf{Physics Inputs Type:}

  \begin{itemize}
  \item
    \emph{Example Instance:}

    \begin{itemize}
    \item
      Initial Positions and Velocities: Player sets the initial launch
      conditions
    \item
      External Forces: None initially
    \end{itemize}
  \end{itemize}
\item
  \textbf{Physics Outputs Type:}

  \begin{itemize}
  \item
    \emph{Example Instance:}

    \begin{itemize}
    \item
      Updated Positions and Velocities: Calculated after each simulation
      step
    \end{itemize}
  \end{itemize}
\end{enumerate}

\textbf{4.2.1.4 User Controls}

\begin{enumerate}
\def\labelenumi{\arabic{enumi}.}
\setcounter{enumi}{8}
\item
  \textbf{User Input Controls Type:}

  \begin{itemize}
  \item
    \emph{Example Instance:}

    \begin{itemize}
    \item
      Launch Angle: Adjustable via touch or drag
    \item
      Force: Adjustable via touch or drag
    \item
      Time Step Size: Fixed at 0.1 seconds
    \end{itemize}
  \end{itemize}
\item
  \textbf{Time Step Size Control Type:}

  \begin{itemize}
  \item
    \emph{Example Instance:}

    \begin{itemize}
    \item
      Time Step Size: Fixed at 0.1 seconds
    \end{itemize}
  \end{itemize}
\end{enumerate}

\textbf{4.2.1.5 Graphics and Sound}

\begin{enumerate}
\def\labelenumi{\arabic{enumi}.}
\setcounter{enumi}{10}
\item
  \textbf{Graphics Type:}

  \begin{itemize}
  \item
    \emph{Example Instance:}

    \begin{itemize}
    \item
      Visual Effects: Cartoonish animations on collision
    \item
      Graphics Rendering Engine: Custom 2D engine
    \end{itemize}
  \end{itemize}
\item
  \textbf{Sound Effects Type:}

  \begin{itemize}
  \item
    \emph{Example Instance:}

    \begin{itemize}
    \item
      Collision Sounds: Comical crashing sounds
    \item
      Launch Sounds: Slingshot stretching and bird squawking
    \end{itemize}
  \end{itemize}
\end{enumerate}

\protect\hypertarget{qrrrrrrrr}{}{}\textbf{4.3 Scope Decisions}

The scope decisions outline the specific boundaries and inclusions for
the development of the physics-based gaming application. These decisions
help define the project's limits and guide the development team
throughout the implementation process.

\textbf{4.3.1 Inclusions}

\begin{enumerate}
\def\labelenumi{\arabic{enumi}.}
\item
  \textbf{Realistic Physics Simulation:}

  \begin{itemize}
  \item
    Inclusion: The application will include a physics engine that
    accurately models collision dynamics, gravitational interactions,
    and additional forces (e.g., wind) to create a realistic gaming
    experience.
  \end{itemize}
\item
  \textbf{Engaging Gameplay:}

  \begin{itemize}
  \item
    Inclusion: The gameplay will focus on challenging players to apply
    physics principles strategically, encouraging experimentation and
    skill development.
  \end{itemize}
\item
  \textbf{User-Friendly Interface:}

  \begin{itemize}
  \item
    Inclusion: The application will provide a user-friendly interface
    allowing players to manipulate initial conditions, adjust
    parameters, and actively participate in the simulation.
  \end{itemize}
\item
  \textbf{Visual and Auditory Feedback:}

  \begin{itemize}
  \item
    Inclusion: Graphics will be used to visually represent physics-based
    interactions, emphasizing rigid body collisions and movements. Sound
    effects corresponding to key physics events will be implemented to
    enhance the overall sensory experience.
  \end{itemize}
\item
  \textbf{Dynamic Challenges:}

  \begin{itemize}
  \item
    Inclusion: Additional forces, such as wind, will be introduced to
    create dynamic challenges, requiring adaptability and skill from the
    players.
  \end{itemize}
\item
  \textbf{Configurability:}

  \begin{itemize}
  \item
    Inclusion: Users will be able to specify initial conditions, launch
    angles, forces, and time step sizes, providing a customizable and
    personalized gaming experience.
  \end{itemize}
\item
  \textbf{Scalability:}

  \begin{itemize}
  \item
    Inclusion: The system will be designed to be extensible for
    potential future expansions, allowing for the addition of new
    features, levels, and challenges.
  \end{itemize}
\end{enumerate}

\protect\hypertarget{qrrrrrrrrrrr}{}{}\textbf{4.4 Modelling Decisions}

In the modelling decisions, various assumptions, theoretical models, and
data definitions are established to guide the development of the
physics-based gaming application.

\protect\hypertarget{qrrrrrrrrrrrr}{}{}\textbf{4.4.1 Assumptions}

\textbf{4.4.1.1 Environment Assumptions}

\begin{enumerate}
\def\labelenumi{\arabic{enumi}.}
\item
  \textbf{Dimensional Constraints:}

  \begin{itemize}
  \item
    Assumption: The gaming environment is strictly two-dimensional,
    simplifying the physics simulation and rendering processes. There is
    no need to account for the complexities of a three-dimensional
    space.
  \end{itemize}
\end{enumerate}

\textbf{4.4.1.2 Gravity Assumptions}

\begin{enumerate}
\def\labelenumi{\arabic{enumi}.}
\setcounter{enumi}{1}
\item
  \textbf{Uniform Gravitational Field:}

  \begin{itemize}
  \item
    Assumption: Gravity acts uniformly on all objects, neglecting
    variations due to altitude, geographic location, or other factors.
    This simplification ensures a consistent and predictable
    gravitational force throughout the game.
  \end{itemize}
\end{enumerate}

\textbf{4.4.1.3 Collision Dynamics Assumptions}

\begin{enumerate}
\def\labelenumi{\arabic{enumi}.}
\setcounter{enumi}{2}
\item
  \textbf{Elastic Collisions:}

  \begin{itemize}
  \item
    Assumption: Collisions between objects are assumed to be perfectly
    elastic, meaning kinetic energy is conserved during interactions.
    This simplification allows for straightforward and predictable
    object responses in the game.
  \end{itemize}
\end{enumerate}

\textbf{4.4.1.4 Wind Force Assumptions}

\begin{enumerate}
\def\labelenumi{\arabic{enumi}.}
\setcounter{enumi}{3}
\item
  \textbf{Constant Wind Forces:}

  \begin{itemize}
  \item
    Assumption: Wind forces acting on objects are considered constant
    within a simulation step, simplifying their application. Dynamic or
    varying wind conditions are not taken into account for simplicity in
    the physics calculations.
  \end{itemize}
\end{enumerate}

\textbf{4.4.1.5 Time Step Assumptions}

\begin{enumerate}
\def\labelenumi{\arabic{enumi}.}
\setcounter{enumi}{4}
\item
  \textbf{Fixed Time Step:}

  \begin{itemize}
  \item
    Assumption: The physics simulation operates on a fixed time step,
    simplifying calculations and ensuring a consistent update rate. This
    assumption may introduce minor inaccuracies but contributes to a
    stable and predictable simulation.
  \end{itemize}
\end{enumerate}

\textbf{4.4.1.6 Object Properties Assumptions}

\begin{enumerate}
\def\labelenumi{\arabic{enumi}.}
\setcounter{enumi}{5}
\item
  \textbf{Homogeneous Object Properties:}

  \begin{itemize}
  \item
    Assumption: Objects within the game environment are considered
    homogeneous in terms of physical properties, such as mass and size.
    This simplification streamlines the physics calculations for various
    object interactions.
  \end{itemize}
\end{enumerate}

\textbf{4.4.1.7 Friction and Air Resistance Assumptions}

\begin{enumerate}
\def\labelenumi{\arabic{enumi}.}
\setcounter{enumi}{6}
\item
  \textbf{Simplified Friction Model:}

  \begin{itemize}
  \item
    Assumption: The friction and air resistance models applied to
    objects are simplified, neglecting certain complexities such as
    surface irregularities. This simplification is made for
    computational efficiency and ease of implementation.
  \end{itemize}
\end{enumerate}

\protect\hypertarget{qrrrrrrrrrrrrrr}{}{}\textbf{4.4.2 Theoretical
Models}

\begin{enumerate}
\def\labelenumi{\arabic{enumi}.}
\item
  \textbf{Projectile Motion Model:}

  \begin{itemize}
  \item
    Model: Utilizes classical projectile motion equations to determine
    the trajectory of the projectile based on initial conditions and
    external forces.
  \end{itemize}
\end{enumerate}

\textbf{Equations:}

\begin{itemize}
\item
  X (t) = x0\hspace{0pt} + v0x \hspace{0pt}t
\item
  y (t) = y0 \hspace{0pt}+ v0y \hspace{0pt} t -- 1/2 gt\^{}2
\end{itemize}

\begin{enumerate}
\def\labelenumi{\arabic{enumi}.}
\setcounter{enumi}{1}
\item
  \textbf{Collision Dynamics Model:}

  \begin{itemize}
  \item
    Model: Employs simplified collision detection and response
    algorithms to handle interactions between rigid bodies.
  \end{itemize}
\end{enumerate}

\textbf{Equations:}

\begin{itemize}
\item
  Collision Detection: Determine if two objects A and B collide.
\item
  Collision Response: Adjust the velocities of colliding objects based
  on the collision type.
\end{itemize}

\begin{enumerate}
\def\labelenumi{\arabic{enumi}.}
\setcounter{enumi}{2}
\item
  \textbf{Gravitational Model:}

  \begin{itemize}
  \item
    Model: Applies Newton's law of universal gravitation to compute the
    gravitational forces acting on each object.
  \end{itemize}
\end{enumerate}

\textbf{Equation:}

\begin{itemize}
\item
  F = Gm1 m2/r\^{}2
\end{itemize}

\begin{enumerate}
\def\labelenumi{\arabic{enumi}.}
\setcounter{enumi}{3}
\item
  \textbf{Wind Force Model:}

  \begin{itemize}
  \item
    Model: Introduces an additional force (wind) with constant magnitude
    and direction to create dynamic challenges.
  \end{itemize}
\end{enumerate}

\textbf{Equation:}

FWind = Magnitude × Direction

\protect\hypertarget{qrrrrrrrrrrrrrrr}{}{}\textbf{4.4.3 General
Definitions}

\begin{enumerate}
\def\labelenumi{\arabic{enumi}.}
\item
  \textbf{Position:}

  \begin{itemize}
  \item
    Definition: The location of an object in the two-dimensional gaming
    environment, represented by coordinates (x, y).
  \end{itemize}
\item
  \textbf{Velocity:}

  \begin{itemize}
  \item
    Definition: The rate of change of an object's position with respect
    to time, represented by components (vx, vy).
  \end{itemize}
\end{enumerate}

4.4.4 Data Definitions

\begin{enumerate}
\def\labelenumi{\arabic{enumi}.}
\item
  \textbf{Projectile Data:}

  \begin{itemize}
  \item
    Definition: Data structure representing a projectile with attributes
    for position, velocity, and mass.
  \end{itemize}
\item
  \textbf{Target/Obstacle Data:}

  \begin{itemize}
  \item
    Definition: Data structure representing a target/obstacle with
    attributes for position, shape, and size.
  \end{itemize}
\end{enumerate}

\protect\hypertarget{qkl}{}{}\textbf{4.4.5 Data Types}


\begin{enumerate}
\def\labelenumi{\arabic{enumi}.}
\item
  \textbf{Position Data Type:}

  \begin{itemize}
  \item
    Type: Double precision floating-point numbers (x, y).
  \end{itemize}
\item
  \textbf{Velocity Data Type:}

  \begin{itemize}
  \item
    Type: Double precision floating-point numbers (vx, vy).
  \end{itemize}
\item
  \textbf{Mass Data Type:}

  \begin{itemize}
  \item
    Type: Double precision floating-point number representing the mass
    of an object.
  \end{itemize}
\end{enumerate}

\protect\hypertarget{qrrrrrrrrrrrrrrrrr}{}{}\textbf{4.4.6 Instance
Models}

\begin{enumerate}
\def\labelenumi{\arabic{enumi}.}
\item
  \textbf{Projectile Instance Model:}

  \begin{itemize}
  \item
    Model: An instance of the projectile data structure with specific
    values for position, velocity, and mass.
  \end{itemize}
\item
  \textbf{Target/Obstacle Instance Model:}

  \begin{itemize}
  \item
    Model: An instance of the target/obstacle data structure with
    specific values for position, shape, and size.
  \end{itemize}
\end{enumerate}

\protect\hypertarget{qrrrrrrrrrrrrrrrrrr}{}{}\textbf{4.4.7 Input Data
Constraints}

\begin{enumerate}
\def\labelenumi{\arabic{enumi}.}
\item
  \textbf{Launch Angle Constraint:}

  \begin{itemize}
  \item
    Constraint: Launch angles are limited to a predefined range (e.g., 0
    to 90 degrees) for realistic projectile motion.
  \end{itemize}
\item
  \textbf{Force Constraint:}

  \begin{itemize}
  \item
    Constraint: Forces applied to projectiles are within a specified
    range to maintain a balanced gaming experience.
  \end{itemize}
\end{enumerate}

\protect\hypertarget{qrrrrrrrrrrrrrrrrrrr}{}{}\textbf{4.4.8 Properties
of a Correct Solution}

\begin{enumerate}
\def\labelenumi{\arabic{enumi}.}
\item
  \textbf{Conservation of Energy:}

  \begin{itemize}
  \item
    Property: The simulation maintains the conservation of energy,
    ensuring that the total energy of the system remains constant.
  \end{itemize}
\end{enumerate}

\textbf{The conservation of energy is expressed by the equation:}

Etotal = Ekinetic + EPotential

Where:

\begin{itemize}
\item
  Etotal is the total energy of the system.
\item
  Ekinetic is the kinetic energy associated with the motion of objects.
\item
  Epotential the potential energy associated with the position of
  objects.
\end{itemize}

\begin{enumerate}
\def\labelenumi{\arabic{enumi}.}
\setcounter{enumi}{1}
\item
  \textbf{Accuracy of Collision Response:}

  \begin{itemize}
  \item
    Property: The collision response accurately reflects the
    conservation of momentum and kinetic energy.
  \end{itemize}
\item
  \textbf{Realistic Projectile Motion:}

  \begin{itemize}
  \item
    Property: The projectile motion adheres to realistic physics
    principles, considering launch angles, initial velocities, and
    gravitational forces.
  \end{itemize}
\item
  \textbf{Dynamic Gameplay:}

  \begin{itemize}
  \item
    Property: The introduction of wind forces creates dynamic and
    challenging gameplay scenarios.
  \end{itemize}
\end{enumerate}

\protect\hypertarget{qb}{}{}\textbf{5. Requirements:}

\protect\hypertarget{qbb}{}{}\textbf{5.1 Functional Requirements}

\textbf{5.1.1 Game Physics}

\begin{enumerate}
\def\labelenumi{\arabic{enumi}.}
\item
  The game shall implement gravity using the formula: \textbf{y =
  y\_initial + v\_initial * t - 0.5 * g * t\^{}2}.
\item
  Horizontal movement shall be implemented using the formula: \textbf{x
  = x\_initial + v\_horizontal * t}.
\item
  Collision detection between objects shall be based on bounding box
  comparison.
\item
  Elastic collision response shall be implemented for basic object
  interaction.
\end{enumerate}

\textbf{5.1.2 Game Elements}

\begin{enumerate}
\def\labelenumi{\arabic{enumi}.}
\item
  The game shall include objects with properties such as position,
  velocity, and dimensions.
\item
  Objects shall have sprite images representing their visual appearance.
\end{enumerate}

\textbf{5.1.3 User Interaction}

\begin{enumerate}
\def\labelenumi{\arabic{enumi}.}
\item
  Players shall interact with the game through controls to trigger
  actions such as launching objects.
\end{enumerate}

\textbf{5.1.4 Use Cases}

Use Case 1: Launching an Object

\textbf{Primary Actor:} Player

\textbf{Description:} The player interacts with the game to launch an
object.

\textbf{Preconditions:}

\begin{itemize}
\item
  The game is in an active state.
\end{itemize}

\textbf{Steps:}

\begin{enumerate}
\def\labelenumi{\arabic{enumi}.}
\item
  The player selects the object to launch.
\item
  The player sets the launch parameters.
\item
  The player triggers the launch action.
\end{enumerate}

\textbf{Postconditions:}

\begin{itemize}
\item
  The launched object follows the specified trajectory.
\end{itemize}

\textbf{Use Case 2: Collision Detection}

\textbf{Primary Actor:} Game Engine

\textbf{Description:} The game engine detects collisions between game
objects.

\textbf{Preconditions:}

\begin{itemize}
\item
  Game objects are in motion.
\end{itemize}

\textbf{Steps:}

\begin{enumerate}
\def\labelenumi{\arabic{enumi}.}
\item
  Continuously monitor the positions of game objects.
\item
  Detect collisions based on bounding box comparison.
\end{enumerate}

\textbf{Postconditions:}

\begin{itemize}
\item
  Collision events trigger appropriate responses in the game.
\end{itemize}

\protect\hypertarget{qbbb}{}{}\textbf{5.2 Non-functional Requirements}

\textbf{5.2.1 Performance}

\begin{enumerate}
\def\labelenumi{\arabic{enumi}.}
\item
  \textbf{Minimum Frame Rate:}

  \begin{itemize}
  \item
    The game shall maintain a minimum frame rate of 30 frames per second
    (FPS) to ensure smooth and enjoyable gameplay.
  \end{itemize}
\item
  Efficient Collision Detection and Response:

  \begin{itemize}
  \item
    Collision detection algorithms and response mechanisms shall be
    optimized to ensure efficient processing and minimal impact on
    gameplay performance.
  \end{itemize}
\end{enumerate}

\textbf{5.2.2 Usability}

\begin{enumerate}
\def\labelenumi{\arabic{enumi}.}
\item
  \textbf{Intuitive User Interface:}

  \begin{itemize}
  \item
    The game shall feature an intuitive and visually appealing user
    interface, ensuring that players can easily navigate through menus
    and interact with game elements.
  \end{itemize}
\item
  \textbf{User-friendly Controls:}

  \begin{itemize}
  \item
    Controls for launching objects and other interactions shall be
    designed to be easy to understand, responsive, and comfortable for
    players to use.
  \end{itemize}
\end{enumerate}

\textbf{5.2.3 Reliability}

\begin{enumerate}
\def\labelenumi{\arabic{enumi}.}
\item
  \textbf{Stability in Physics Calculations:}

  \begin{itemize}
  \item
    The game's physics calculations, including gravity and object
    interactions, shall be implemented in a stable and reliable manner
    to minimize bugs and unexpected behaviors.
  \end{itemize}
\item
  \textbf{Robust Collision Handling:}

  \begin{itemize}
  \item
    The collision handling system shall be robust, capable of accurately
    detecting collisions and responding appropriately to ensure the
    integrity of the game's physics simulation.
  \end{itemize}
\end{enumerate}

\textbf{Non-functional Requirements Physics Simulation}

\begin{enumerate}
\def\labelenumi{\arabic{enumi}.}
\item
  \textbf{Realism and Accuracy:}

  \begin{itemize}
  \item
    The physics simulation shall aim for a balance between realism and
    gameplay, ensuring that the behavior of game objects aligns with
    real-world physics principles.
  \item
    Deviations from real-world physics shall be intentional and
    communicated clearly to players.
  \end{itemize}
\item
  \textbf{Stability in Complex Scenes:}

  \begin{itemize}
  \item
    The physics engine shall be capable of maintaining stability and
    accuracy even in scenes with a high number of interacting objects or
    complex geometric shapes.
  \end{itemize}
\item
  \textbf{Dynamic Time Step:}

  \begin{itemize}
  \item
    The physics simulation shall incorporate a dynamic time step
    mechanism to adapt to varying processing speeds, maintaining
    stability and accuracy across different hardware configurations.
  \end{itemize}
\end{enumerate}

\textbf{5.2.5 Object Interactions}

\begin{enumerate}
\def\labelenumi{\arabic{enumi}.}
\item
  \textbf{Elasticity Consistency:}

  \begin{itemize}
  \item
    Objects in the game shall exhibit consistent elasticity in
    collisions, providing predictable reactions based on their physical
    properties.
  \end{itemize}
\item
  \textbf{Friction and Drag:}

  \begin{itemize}
  \item
    The game shall simulate realistic friction and air resistance,
    affecting the motion of objects in a manner that aligns with
    real-world physics.
  \end{itemize}
\item
  \textbf{Object Material Properties:}

  \begin{itemize}
  \item
    Game objects shall have customizable material properties (e.g.,
    density, elasticity, friction) to allow developers to fine-tune the
    physics simulation for specific gameplay scenarios.
  \end{itemize}
\end{enumerate}

\textbf{5.2.6 Debugging and Analysis Tools}

\begin{enumerate}
\def\labelenumi{\arabic{enumi}.}
\item
  \textbf{Physics Debugging Tools:}

  \begin{itemize}
  \item
    The game shall include built-in tools for debugging and visualizing
    physics interactions, helping developers identify and address issues
    related to object movements, collisions, and forces.
  \end{itemize}
\item
  \textbf{Performance Profiling:}

  \begin{itemize}
  \item
    Tools for performance profiling of the physics engine shall be
    available, allowing developers to identify bottlenecks and optimize
    the simulation for better performance.
  \end{itemize}
\end{enumerate}

\textbf{5.2.7 Adaptability and Extensibility}

\begin{enumerate}
\def\labelenumi{\arabic{enumi}.}
\item
  \textbf{Modular Physics Components:}

  \begin{itemize}
  \item
    The physics system shall be designed with modularity in mind,
    allowing for easy integration of additional physics features or
    third-party physics libraries.
  \end{itemize}
\item
  \textbf{Parameter Tuning:}

  \begin{itemize}
  \item
    The game shall provide parameters that can be tuned to adjust the
    overall physics behavior, catering to different game genres or
    design preferences.
  \end{itemize}
\end{enumerate}

\protect\hypertarget{qn}{}{}\textbf{6. Likely Changes:}

\textbf{6.1 Physics Model Refinement:}

Recognize the possibility of refining the physics model to enhance
realism based on player feedback or advancements in simulation
techniques.

\textbf{6.2 Additional Forces:}

Anticipate the introduction of new forces or environmental factors for
added complexity and variety in gameplay.

\textbf{6.3 User Interface Enhancements:}

Acknowledge potential adjustments to the user interface to improve user
experience and accommodate additional controls.

\protect\hypertarget{qm}{}{}\textbf{7. Unlikely Changes:}

\textbf{7.1 Core Physics Algorithms:}

Specify that fundamental physics algorithms governing collisions,
gravitational interactions, and rigid body dynamics are unlikely to
change unless there are groundbreaking advancements in simulation
techniques.

\textbf{7.2 Rigid Body Assumption:}

Clarify that the assumption of rigid bodies is a foundational element
and is unlikely to change, ensuring stability in the physics simulation.

\textbf{7.3 Simulation Framework:}

Highlight that the overall simulation framework, once established, is
unlikely to undergo significant changes, providing stability for
integration with other game
elements.\protect\hypertarget{_Hlk157473991}{}{}

\textbf{8. Traceability Matrix}

A traceability matrix and graphs are typically tools used in project
management and software development to ensure that requirements,
features, and decisions can be traced back to their origins and that
there is a clear mapping between various elements of the project. In our
case, let's focus on creating a simplified traceability matrix.

\textbf{Traceability Matrix :}

\begin{longtable}[]{@{}
  >{\raggedright\arraybackslash}p{(\columnwidth - 6\tabcolsep) * \real{0.2500}}
  >{\raggedright\arraybackslash}p{(\columnwidth - 6\tabcolsep) * \real{0.2500}}
  >{\raggedright\arraybackslash}p{(\columnwidth - 6\tabcolsep) * \real{0.2500}}
  >{\raggedright\arraybackslash}p{(\columnwidth - 6\tabcolsep) * \real{0.2500}}@{}}
\toprule
\begin{minipage}[b]{\linewidth}\raggedright
\textbf{Requirement}

\textbf{ID}
\end{minipage} & \begin{minipage}[b]{\linewidth}\raggedright
\textbf{4.2.1 Types}

\textbf{1}
\end{minipage} & \begin{minipage}[b]{\linewidth}\raggedright
\textbf{4.2.2 Solution}

\textbf{Characteristics}

\textbf{Specification}
\end{minipage} & \begin{minipage}[b]{\linewidth}\raggedright
\textbf{4.4 Modelling}

\textbf{Decisions}
\end{minipage} \\
\midrule
\endhead
1 & Projectile Type & Game Elements & Assumptions \\
2 & Targets/Obstacles type & Game Elements & Assumptions \\
3 & Collision Dynamics type & Physics Simulation & Assumptions,

Theoretical Models \\
4 & Gravitational Forces type & Physics Simulation & Assumptions,

Theoretical Models \\
5 & Additional Forces Type

(e.g., Wind) & Physics Simulation & Assumptions,

Theoretical Models \\
6 & Physics Engine Type & Physics Simulation & Assumptions,

Theoretical Models \\
7 & Physics Inputs Type & Physics Inputs & Assumptions,

Theoretical Models \\
8 & Physics outputs Type & Physics 0utputs & Assumptions,

Theoretical Models \\
9 & User Input Controls Type & Controls & Assumptions,

Theoretical Models \\
10 & Time Step Size Control type & Controls & Assumptions,

Theoretical Models \\
11 & Graphics Type & Graphics and Sound & Assumptions,

Theoretical Models \\
12 & Sound Effects Type & Graphics and Sound & Assumptions,

Theoretical Models \\
\bottomrule
\end{longtable}

\textbf{Models to Code}

\begin{longtable}[]{@{}
  >{\raggedright\arraybackslash}p{(\columnwidth - 2\tabcolsep) * \real{0.5000}}
  >{\raggedright\arraybackslash}p{(\columnwidth - 2\tabcolsep) * \real{0.5000}}@{}}
\toprule
\begin{minipage}[b]{\linewidth}\raggedright
\textbf{Model}
\end{minipage} & \begin{minipage}[b]{\linewidth}\raggedright
\textbf{Code Functions/Structures}
\end{minipage} \\
\midrule
\endhead
Projectile Type & Bird struct \\
Targets/Obstacles Type & Pig struct \\
Collision Dynamics Type & checkCollision() function \\
Gravitational Forces Type & updateBird Position() function \\
Additional Forces Type (e.g., Wind) & updateBird Position() function \\
Physics Engine Type & updateBird Position() function \\
Physics Inputs Type & handleInput() function \\
Physics outputs Type & updateBird Position() function, renderGame()
function \\
User Input Controls type & handleInput() function \\
Time Step Size Control Type & Main loop with time step \\
Graphics Type & renderGame() function \\
Sound Effects Type & renderGame() function \\
\bottomrule
\end{longtable}

This directed acyclic graph illustrates the dependencies between
different components of the project. For example, the "Physics
Simulation" node depends on "Game Elements," "Controls," and "Graphics
and Sound." This graph helps visualize the flow of dependencies.

\textbf{Requirement Traceability Graph}

This graph shows the relationships between different project elements,
starting from the requirements. It demonstrates how requirements are
connected to different types, models, and eventually, the code
components.

\protect\hypertarget{qzz}{}{}\textbf{9. Development Plan}

A development plan outlines the steps and activities required to achieve
the goals of a project. Here's a simplified development plan for the
physics-based gaming application:

1. \textbf{Project Kickoff (Week 1-2)}

\begin{itemize}
\item
  \textbf{Objectives:}

  \begin{itemize}
  \item
    Clarify project scope, goals, and requirements.
  \item
    Assemble development team.
  \item
    Setup project repository and version control.
  \end{itemize}
\item
  \textbf{Tasks:}

  \begin{itemize}
  \item
    Hold a kickoff meeting.
  \item
    Define roles and responsibilities.
  \item
    Create and share project documentation.
  \item
    Establish version control using Git.
  \end{itemize}
\end{itemize}

2. \textbf{Requirement Analysis (Week 3-4)}

\begin{itemize}
\item
  \textbf{Objectives:}

  \begin{itemize}
  \item
    Refine and clarify project requirements.
  \item
    Prioritize features and functionalities.
  \item
    Finalize the system requirements specification (SRS).
  \end{itemize}
\item
  \textbf{Tasks:}

  \begin{itemize}
  \item
    Conduct meetings with stakeholders.
  \item
    Revise and update SRS document.
  \item
    Prioritize features based on importance and complexity.
  \end{itemize}
\end{itemize}

3. \textbf{Design and Architecture (Week 5-8)}

\begin{itemize}
\item
  \textbf{Objectives:}

  \begin{itemize}
  \item
    Define system architecture.
  \item
    Create class diagrams and data flow diagrams.
  \item
    Select appropriate frameworks and libraries.
  \end{itemize}
\item
  \textbf{Tasks:}

  \begin{itemize}
  \item
    Develop a high-level architecture plan.
  \item
    Create detailed class diagrams for key components.
  \item
    Choose graphics and sound libraries/frameworks.
  \item
    Decide on the overall system structure.
  \end{itemize}
\end{itemize}

4. \textbf{Implementation (Week 9-12)}

\begin{itemize}
\item
  \textbf{Objectives:}

  \begin{itemize}
  \item
    Build the core functionalities of the game.
  \item
    Develop physics engine, collision detection, and rendering.
  \item
    Implement user controls and input handling.
  \end{itemize}
\item
  \textbf{Tasks:}

  \begin{itemize}
  \item
    Begin coding the physics engine.
  \item
    Implement basic game rendering.
  \item
    Develop user input controls.
  \item
    Integrate collision detection algorithms.
  \end{itemize}
\end{itemize}

5. \textbf{Testing and Debugging (Week 13-16)}

\begin{itemize}
\item
  \textbf{Objectives:}

  \begin{itemize}
  \item
    Identify and fix software defects.
  \item
    Conduct unit testing and integration testing.
  \item
    Ensure physics simulations are accurate.
  \end{itemize}
\item
  \textbf{Tasks:}

  \begin{itemize}
  \item
    Perform unit tests on individual components.
  \item
    Conduct integration tests for system interactions.
  \item
    Debug and resolve any issues identified during testing.
  \end{itemize}
\end{itemize}

6. \textbf{Refinement and Optimization (Week 17-18)}

\begin{itemize}
\item
  \textbf{Objectives:}

  \begin{itemize}
  \item
    Optimize code for performance.
  \item
    Refine user interface and graphics.
  \item
    Enhance overall user experience.
  \end{itemize}
\item
  \textbf{Tasks:}

  \begin{itemize}
  \item
    Optimize algorithms for computational efficiency.
  \item
    Improve graphics rendering.
  \item
    Gather user feedback for UI/UX improvements.
  \end{itemize}
\end{itemize}

7. \textbf{Documentation (Week 19-20)}

\begin{itemize}
\item
  \textbf{Objectives:}

  \begin{itemize}
  \item
    Document code, design decisions, and usage instructions.
  \item
    Finalize system documentation.
  \end{itemize}
\item
  \textbf{Tasks:}

  \begin{itemize}
  \item
    Write code comments for clarity.
  \item
    Update design documentation.
  \item
    Create a user guide for the application.
  \end{itemize}
\end{itemize}

8. \textbf{Deployment (Week 21-22)}

\begin{itemize}
\item
  \textbf{Objectives:}

  \begin{itemize}
  \item
    Prepare the application for release.
  \item
    Set up distribution channels (if applicable).
  \end{itemize}
\item
  \textbf{Tasks:}

  \begin{itemize}
  \item
    Package the application for deployment.
  \item
    Create installers or deployment scripts.
  \item
    Publish the application on relevant platforms.
  \end{itemize}
\end{itemize}

9. \textbf{Post-Launch Support and Updates (Ongoing)}

\begin{itemize}
\item
  \textbf{Objectives:}

  \begin{itemize}
  \item
    Address user feedback and bug reports.
  \item
    Implement updates and improvements.
  \end{itemize}
\item
  \textbf{Tasks:}

  \begin{itemize}
  \item
    Monitor user feedback channels.
  \item
    Prioritize and address reported issues.
  \item
    Release updates as needed.
  \end{itemize}
\end{itemize}

\protect\hypertarget{qzzz}{}{}\textbf{10. Physical Constants}

Physical constants are fundamental values used to model real-world
physics within the game. They contribute to the accuracy and
authenticity of the in-game physics simulation.

\textbf{Gravity Constant (\emph{g})}

\begin{itemize}
\item
  Standard Earth's gravity = 9.81m/s\^{}2.
\end{itemize}

\textbf{Air Density (\emph{ρ})}

\begin{itemize}
\item
  Average air density at sea level: 1.225kg/m\^{}3.
\end{itemize}

\textbf{Friction Coefficient (\emph{μ})}

\begin{itemize}
\item
  Coefficient of friction for surfaces: Typically, between 0 and 1,
  representing different friction levels.
\end{itemize}

\textbf{Game Parameters}

Game parameters are constants that define specific aspects of the gaming
experience, such as scoring, difficulty, and user interactions.

\begin{enumerate}
\def\labelenumi{\arabic{enumi}.}
\item
  \textbf{Score Multiplier (SCORE\_MULTIPLIERSCORE)}

  \begin{itemize}
  \item
    A factor that multiplies the base score for achieving certain
    in-game objectives.
  \end{itemize}
\item
  \textbf{Difficulty Threshold (DIFFICULTY\_THRESHOLD)}

  \begin{itemize}
  \item
    A value that determines the difficulty level, affecting factors like
    enemy behavior, speed, or obstacle complexity.
  \end{itemize}
\item
  \textbf{Time Step (Δ\emph{t})}

  \begin{itemize}
  \item
    The time step used in the physics simulation, influencing the
    accuracy of the simulation. Smaller values result in more accurate
    but computationally intensive simulations.
  \end{itemize}
\end{enumerate}

\textbf{Environmental Constants}

Environmental constants define features related to the virtual
environment in which the game takes place. These constants contribute to
the atmosphere and ambiance of the gaming world.

\begin{enumerate}
\def\labelenumi{\arabic{enumi}.}
\item
  \textbf{Wind Speed (WIND\_SPEED)}

  \begin{itemize}
  \item
    The speed of the virtual wind, affecting the trajectory of
    projectiles or objects in motion.
  \end{itemize}
\item
  \textbf{Ambient Light Intensity (AMBIENT\_LIGHT)}

  \begin{itemize}
  \item
    The intensity of the ambient light in the virtual environment,
    influencing the overall visual atmosphere of the game.
  \end{itemize}
\item
  \textbf{Background Music Volume (MUSIC\_VOLUME)}

  \begin{itemize}
  \item
    The volume level of the background music, contributing to the
    auditory experience of the game.
  \end{itemize}
\end{enumerate}

\protect\hypertarget{qop}{}{}\textbf{11. Project Goal}

The primary ambition of this project is to achieve a breakthrough in
physics-based computation within the gaming realm. By focusing on the
intricacies of realistic collision dynamics, gravitational interactions,
and pioneering the implementation of dynamic momentum disruptions, the
project aims to elevate the gaming experience to a level where players
actively engage with and manipulate fundamental physics principles. When
a bird collides, the goal is to authentically break the gravity
momentum, simulating a quantum leap in computational realism. This
project aspires to contribute to the advancement of physics
understanding in the gaming community, pushing the boundaries of what is
achievable in virtual worlds through cutting-edge computational physics.

\textbf{Realistic Physics Simulation:}

Implement a physics engine that accurately models collision dynamics,
gravitational interactions, and additional forces to create a realistic
and immersive gaming experience.

\textbf{Engaging Gameplay:}

Design a gaming environment that challenges players to apply physics
principles strategically, encouraging experimentation and skill
development.

\textbf{User-Friendly Interface:}

Provide a user-friendly interface allowing players to manipulate initial
conditions, adjust parameters, and actively participate in the
simulation.

\textbf{Visual and Auditory Feedback:}

Utilize graphics to visually represent physics-based interactions,
emphasizing rigid body collisions and movements. Implement sound effects
corresponding to key physics events, enhancing the overall sensory
experience.

\textbf{Dynamic Challenges:}

Introduce variability through additional forces like wind, creating
dynamic challenges that require adaptability and skill from the players.

\textbf{Configurability:}

Allow users to specify initial conditions, launch angles, forces, and
time step sizes, providing a customizable and personalized gaming
experience.

\textbf{Scalability:}

Design the system to be extensible for potential future expansions,
allowing for the addition of new features, levels, and challenges.

\textbf{Achieve Physics/Computing Breakthrough:}

Develop a physics-based gaming application that explores breakthroughs
in physics computation, emphasizing the accurate representation of
momentum changes during collisions. Strive for a level of computational
fidelity that contributes to advancements in real-time physics
simulations within gaming environments.

\protect\hypertarget{qbn}{}{}\textbf{12. Goal in Physics}

\textbf{Achieving in Quantum Precision Physics-Based Gaming: Unleashing
Realistic Momentum Dynamics}

The paramount goal of this project is to attain a pinnacle of precision
in physics-based gaming. By intricately modeling realistic collision
dynamics, gravitational interactions, and introducing innovative
momentum dynamics, the aim is to achieve a gaming experience that
mirrors the precision of quantum physics principles. When a bird
collides, pushing the boundaries of what is achievable in virtual
physics simulations. This project is dedicated to advancing the
understanding of physics principles in gaming, offering players an
immersive experience that reflects the intricacies and precision of
real-world physics.

\end{document}
